%\section{Motivation}
%\subsection{Value of accurate rainfall-runoff modelling}
Rainfall-runoff modelling is a very important problem for scientists and 
companies in the field of hydrology.
Results found by \cite{lstm_first_paper} indicate that given a good dataset,
recurrent neural networks in the form of Long Short-Term Memory cells (LSTMs)
can give results equal in quality to more classical hydrological models.
This was an important step towards data driven hydrological modelling. 

Today there exist several well working traditional physical models for this usage. \citationneeded
A drawback of these models is that they have a lot of parameters that havbe to be 
calibrated for each basin individually. They lack generalisation because of a lack of 
physical information in the models. \citationneeded In 2019 Kratzert et. al. showed
that we can use Long Short-Term Memory (LSTM) models to model several basins with the same model
given enough static information about the basins \cite{lstm_second_paper}. These 
static features are things like forest coverage, soil type, etc. While these LSTM 
models did not outperform traditional models when calibrated on a single basin at 
a time, they did outperform traditional models when used for generalisation.

In addition to the benefits of LSTM models performing well on several basins, 
one of the most sought after concepts in the field of hydrology right now is the 
ability to do predictions on basins where no prior streamflow data is available. \citationneeded
This would grant companies the ability to model older basins without proper sensors
and would allow companies to model the basins of their competitors to better price 
their produced electricity as a response to this. There has been made some initial 
progress on trying LSTM models for this use case showing promising results \cite{lstm_third_paper}.
While Kratzert et. al. showed that this works well on a single dataset, namely the 
CAMELS US dataset \cite{CAMELS_US}, it is important to see what results one may get 
also on different datasets. In this thesis we will mainly forcus doing prediction on 
the CAMELS GB dataset \cite{CAMELS_GB} in addition to data derived from Norwegian basins provided by 
Statkraft.

\section{A brief history of traditional rainfall-runoff modelling}

\section{History of the Long Short-Term Memory model}
\subsection{Early history}
The concept of neural networks has existed for decades, the earliest example being 
from 1958 \cite{rosenblatt1958perceptron}. The standard fully connected feed forward 
neural network is unable to learn dependencies over time, which is why the Recurrent 
Neural Network (RNN) was invented. \citationneeded
Due to several drawbacks of the default RNN architecture several new RNN-types 
were created. The two most well known are the GRU and LSTM models. \citationneeded
\subsection{Recent developements and the resurgence of machine learning}
\subsection{Usage in rainfall-runoff modelling}

\section{Goals}
Riding on the promising results from \cite{lstm_second_paper} and \cite{lstm_third_paper}
we want to see if LSTM models are generally applicable for rainfall-runoff modelling 
and not just in the special case of the CAMELS US \cite{CAMELS_US} dataset. This 
is important as one of the main drawbacks to the machine learning approach is that 
it is difficult if not impossible to mathematically prove your results, meaning 
we have to approach the issue with extensive imperic research preferably on as 
many datasts as possible. 
\section{Our contribution}
We show that there is potential for more accurate general physical models based on
available data that is not taken into account by traditional models.

\section{Thesis structure}
This thesis is divided into 7 chapters, most chapters also being sub-divided into
sections. 
\begin{enumerate}
\item  The introduction. Here we explain the importance of this work and briefly summarize what has been done by others before us.
\item Theory. This chapter is divided into two parts:
    \begin{enumerate}
    \item Rainfall-runoff modelling. This section describes the relevant theory for understanding the physics captured in today's popular rainfall-runoff models. We present several 
    \item Machine learning. Here we try to give a thorough overview of the concepts needed to understand the models and methods we apply in this thesis. We give a brief mathematical and physical explanation of the model, and try to present physical effects that are parametrized away instead of being properly modeled. This is important for the goal of this thesis.
    \end{enumerate}
    \item Data. Here we explain and analyze several aspects of the dataset(s???) \citationneeded  we use. This gives important context for why and how we employ the methods described in the method chapter.
    \item Method. Here we describe in detail how we structure our model, what programming frameworks we use and how we analyze our results. Brief documentation  and a minimal running example of the camelsml Python-package are also presented.
    \item Results. Here we present our most interesting results and give initial comments.
    \item Discussion. This chapter presents analysis of our results.
    \item Conclusion. A short summary of our most interesting findings along with what we feel will be a logical next step for further research.
In addition the thesis also has an appendix which presents longer mathematical 
calculations which will be referenced to in the main body of the thesis. 

\end{enumerate}
\nocite{4160265}
\nocite{mckinney-proc-scipy-2010}
\nocite{2020NumPy-Array}
\nocite{2020SciPy-NMeth}
