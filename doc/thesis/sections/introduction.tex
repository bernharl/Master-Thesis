%\section{Motivation}
%\subsection{Value of accurate rainfall-runoff modelling}
Rainfall-runoff modelling is a very important problem for scientists and 
companies in the field of hydrology. Improving the prediction of runoff/streamflow 
based on historic data could lead to more accurate prediction of power generation 
in hydro power plants, flood forecasting and general water resource management.

Today there exists several models for this usage, some with less focus on physics: 
conceptual models, and some with a larger focus on physics: process-driven models. 
A drawback of these models is that they have many parameters that have to be 
calibrated for each basin individually \citep{VICbench, BiasVarianceVIC, lstm_third_paper}. 
This leads to the models failing to generalize, as well as leading to the 
reduction of physical interpetability. Process-driven models with physical interpretations 
of their parameters have been shown to perform better once one forgoes the expected 
values of said parameters, and instead callibrate these as well \citep[e.g.][]{VICbench, BiasVarianceVIC}.
This is mostly due to the fact that the models are are not purely based on the 
underlying physics in a system, such as for conceptual models, and due to 
the otherwise excessive need to do on-site data collection for use in more heavily 
process-driven models. 

With the explosive increase of observational data in recent years (mostly from 
improvement in sensor and satellite techniques) and the increase in computational 
power, new methods for streamflow forecasting can be explored. The new data could 
lessen the need to do on-site probing to model the complex physical system that is 
subsurface flow. Implementing new data in existing mdoels is hard, a simpler 
way to implement it is instead to use purely data-driven models. Machine learning 
models have shown potential for predicting on catchments described in the CAMELS 
dataset \citep{lstm_first_paper, CAMELS_US}. The potential for using this dataset in addition to several, 
newer datasets from other regions of the world to train generalizable 
machine learning models is higher. Being able to perform ungauged prediction 
(predictions on a basin without callibrating on it) could yield the benefit of 
getting approximate prediction of a catchment where no streamflow measurements 
have been performed.

The downside of using machine learning models (as well as other purely data-driven 
models) is that the models are less interpretable. Even conceptual models, being 
less physically based than process-driven models, are somewhat consistent with 
how a catchment works in the real world. This, for instance, makes it easier to 
discover when a model is predicting non-physicaly consistent results, as one 
can look at the outcome of each individual process in the model. For machine learning 
models, there is no such structure, making this a much harder task.
Despite this, machine learning models 
can still be used to gain an understanding of which attributes contained in 
large scale datasets carry the most information, therefore narrowing down the 
process of using them to improve process-driven models.
\section{Goals}
There are two main goals in this thesis: Analysing static attributes to get a 
further indication of which (if any) attributes can be used to improving the 
understanding of the underlying physics in rainfall-runoff modelling, as well 
as discovering whether existing machine learning models can generalize across 
more than one dataset spanning several regions. Attribute importance ranking 
across several datasets is also of interest.

\section{Our contribution}
We show that our model performs better using the extra information in the static 
catchment features in the CAMELS-GB \citep{CAMELS_GB} dataset than if it were trained 
only on timeseries, as well as being able to perform well on ungauged basins.

We are able to do an analysis on the importance of these attributes and give a brief 
indication where to begin using these attributes to improve conceptual and process-driven 
models.

For further analysis we also provide the main bulk of 
the code used in this thesis as a python package released under the Apache 2.0 
license. This code is originally based on the code released by \citet{lstm_second_paper}.

\section{Thesis structure}
This thesis is divided into 7 chapters, most chapters also being sub-divided into
sections. 
\begin{enumerate}
\item  The introduction. Here we explain the importance of this work and briefly summarize what has been done by others before us.
\item Theory. This chapter is divided into two parts:
    \begin{enumerate}
    \item Rainfall-runoff modelling. This section describes the relevant theory 
        for understanding the physics captured in today's popular rainfall-runoff 
            models. We give a brief overview of a popular conceptual model known 
            as SACramento Soil Moisture Accounting (SAC-SMA), and 
            the process driven models Variable Infiltration Capacity (VIC) and 
            National Water Model (NWM), as well as drawbacks and limitations 
            of these models. 
    \item Machine learning. Here we try to give a overview of the concepts 
        needed to understand the machine learning models we employ in this thesis. 
            We also give physics-based arguments for why the Long Short-Term Memory 
            (LSTM) model is a good fit for hydrological modelling.
    \end{enumerate}
    \item Data. Here we give a brief explanation of how the CAMELS and CAMELS-GB 
    datasets are structured so that the Method and Results chapters are more easily followed.
    \item Method. Here we describe in detail how we structure our model, what programming 
        frameworks we use and how we analyse our results. An explanation of the 
        code used to derive our results is also included.
    \item Results. Here we present our most interesting results and state initial 
        observations.
    \item Discussion. This chapter presents analysis of our results. We try to 
        find potential ways to connect our results to the goals of this thesis.
    \item Conclusion. A short summary of our most interesting findings along with 
        what we feel could be logical next steps for further research.
\end{enumerate}
In addition, the appendix contains documentation of the code used in this thesis.
\nocite{4160265}
\nocite{mckinney-proc-scipy-2010}
\nocite{2020NumPy-Array}
\nocite{2020SciPy-NMeth}
\nocite{NEURIPS2019_9015}
\nocite{Prechelt1998}
