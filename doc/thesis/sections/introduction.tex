\section{Motivation}
\subsection{Value of accurate rainfall-runoff modelling}
Rainfall-runoff modelling is a very important problem for scientists and 
companies in the field of hydrology.
Results found by \cite{lstm_first_paper} indicate that given a good dataset,
recurrent neural networks in the form of Long Short-Term Memory cells (LSTMs)
can give results equal in quality to more classical hydrological models.
This was an important step towards data driven hydrological modelling. 

Today there exist several well working traditional physical models for this usage. \citationneeded
A drawback of these models is that they have a lot of parameters that havbe to be 
calibrated for each basin individually. They lack generalisation because of a lack of 
physical information in the models. \citationneeded In 2019 Kratzert et. al. showed
that we can use Long Short-Term Memory (LSTM) models to model several basins with the same model
given enough static information about the basins \cite{lstm_second_paper}. These 
static features are things like forest coverage, soil type, etc. While these LSTM 
models did not outperform traditional models when calibrated on a single basin at 
a time, they did outperform traditional models when used for generalisation.

In addition to the benefits of LSTM models performing well on several basins, 
one of the most sought after concepts in the field of hydrology right now is the 
ability to do predictions on basins where no prior streamflow data is available. 
This would grant companies the ability to model older basins without proper sensors
and would allow companies to model the basins of their competitors to better price 
their produced electricity as a response to this. There has been made some initial 
progress on trying LSTM models for this use case showing promising results \cite{lstm_third_paper}.
While Kratzert et. al. showed that this works well on a single dataset, namely the 
CAMELS US dataset \cite{CAMELS_US}, it is important to see what results one may get 
also on different datasets. In this thesis we will mainly forcus doing prediction on 
the CAMELS GB dataset \cite{CAMELS_GB} in addition to data derived from Norwegian basins provided by 
Statkraft.
\section{History of the Long Short-Term Memory model}
\subsection{Early history}
The concept of neural networks has existed for decades, the earliest example being 
from 1958 \cite{rosenblatt1958perceptron}. The standard fully connected feed forward 
neural network is unable to learn dependencies over time, which is why the Recurrent 
Neural Network (RNN) was invented. \citationneeded
Due to several drawbacks of the default RNN architecture several new RNN-types 
were created. The two most well known are the GRU and LSTM models. \citationneeded
