%\section{Motivation}
%\subsection{Value of accurate rainfall-runoff modelling}
Rainfall-runoff modelling is a very important problem for scientists and 
companies in the field of hydrology.
Results found by \cite{lstm_first_paper} indicate that given a good dataset,
recurrent neural networks in the form of Long Short-Term Memory cells (LSTMs)
can give results equal in quality to more classical hydrological models.
This was an important step towards data driven hydrological modelling. 

Today there exist several well working traditional physical models for this usage. \citationneeded
A drawback of these models is that they have a lot of parameters that havbe to be 
calibrated for each basin individually. They lack generalisation because of a lack of 
physical information in the models. \citationneeded In 2019 Kratzert et. al. showed
that we can use Long Short-Term Memory (LSTM) models to model several basins with the same model
given enough static information about the basins \cite{lstm_second_paper}. These 
static features are easily obtainable through for example observational satelite 
data \cite{CAMELS_US}. While these LSTM models did not outperform traditional models when calibrated on a single basin at 
a time, they did outperform traditional models when used for generalisation.

In addition to the benefits of LSTM models performing well on several basins, 
one of the most sought after concepts in the field of hydrology right now is the 
ability to do predictions on basins where no prior streamflow data is available. \citationneeded
This would grant companies the ability to model older basins without proper sensors
and would allow companies to model the basins of their competitors to better price 
their produced electricity as a response to this. There has been made some initial 
progress on trying LSTM models for this use case showing promising results \cite{lstm_third_paper}.

\section{Goals}
Our main goal of this thesis is to improve the understanding of the underlying 
physics in rainfall-runoff modelling. While we already know that \cite{lstm_third_paper} 
shows promising results from using a purely data-driven model, but there are several 
disadvantages of using so called "black box" models. What we therefore are interested 
in is to show that such a black box model can be used for analysis of a dataset 
to give an indication on the best way forward to include the easily obtained 
static data included in the CAMELS \cite{CAMELS_US} and CAMELS-GB \cite{CAMELS_GB} 
datasets. 
While previous work has been focusing on the 
CAMELS dataset \cite{CAMELS_US}, it is important to see what results one may get 
also on different datasets. In this thesis we will mainly focus doing prediction on 
the CAMELS GB dataset \cite{CAMELS_GB}. This way even if our attempts at finding 
relevant physical information in the static features we will at least have shown 
that the results of \cite{lstm_third_paper} also seem to apply on a separate dataset 
gathered from a different place in the world. 
\section{Our contribution}
We show that our model performs better using the extra information in the static 
catchment features in the CAMELS-GB dataset than if it were trained only on timeseries.
We are able to do an analysis on the importance of these features and give a brief 
indication where to begin using these features for better calibration of physics-
based traditional models. For further analysis we also provide the main bulk of 
the code used in this thesis as a python package released under the Apache 2.0 
license. This code is originally based on the code used in \cite{lstm_second_paper}.
\section{Thesis structure}
This thesis is divided into 7 chapters, most chapters also being sub-divided into
sections. 
\begin{enumerate}
\item  The introduction. Here we explain the importance of this work and briefly summarize what has been done by others before us.
\item Theory. This chapter is divided into two parts:
    \begin{enumerate}
    \item Rainfall-runoff modelling. This section describes the relevant theory for understanding the physics captured in today's popular rainfall-runoff models. We present a popular traditional model known as SAC-SMA an show parts of it that may be improved using the insight gathered in this thesis work.
    \item Machine learning. Here we try to give a thorough overview of the concepts needed to understand the machine learning models we employ in this work. We also give physics-based arguments for why the Long Short-Term Memory (LSTM) model is a good fit for this kind of physical system.
    \end{enumerate}
    \item Data. Here we give a brief explanation of how the CAMELS and CAMELS-GB 
    datasets are structured so that the Method and Results chapters is more easily followed.
    \item Method. Here we describe in detail how we structure our model, what programming frameworks we use and how we analyze our results. An explanation of the code used to derive our results is also included.
    \item Results. Here we present our most interesting results and give initial comments.
    \item Discussion. This chapter presents analysis of our results. We try to find potential ways to connect our results to the goals of this thesis.
    \item Conclusion. A short summary of our most interesting findings along with what we feel will be a logical next step for further research.
\end{enumerate}
In addition the thesis includes a couple more details in the appendix, such as 
minimal running examples of the CamelsML python package along with documentation.
\nocite{4160265}
\nocite{mckinney-proc-scipy-2010}
\nocite{2020NumPy-Array}
\nocite{2020SciPy-NMeth}
\nocite{NEURIPS2019_9015}
\nocite{Prechelt1998}
\nocite{elemstatlearn}
