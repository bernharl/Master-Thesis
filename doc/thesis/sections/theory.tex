\section{Rainfall-Runoff modelling}
\subsection{Physical modelling}
\subsection{Drawbacks}

\section{Machine Learning}
We give a brief explanation of the basics in Machine Learning here for better context 
before elaborating on the LSTM model central to this thesis.
The term Machine Learning was coined in \citationneeded by .... 
Machine Learning is a type of frequentist approach to statistical analysis where 
one creates a statistical model, often with several million parameters and finds 
the value of each parameter that makes the model approximate the data in the most 
accurate manner. How to find these parameters and how they are used differ for each 
model type. 
\subsection{Linear regression}
In the simple case of the Ordinary Least Squares (OLS) model we have a model on the form 
\begin{equation}
\mathbf{\hat{y}} = \mathbf{\beta} \mathbf{X}
\label{OLS}
\end{equation}
This assumes that the outcome $\mathbf{\hat{y}}$ can we represented as a linear combination 
of some fitted parameters $\mathbf{\beta}$ and the input features $\mathbf{X}$.
The goal here is then to find the minimum of the mean squared error (MSE) of this.
The MSE is defined as 
\begin{equation}
MSE = |\mathbf{y} - \mathbf{\hat{y}}|^2
\label{MSE}
\end{equation}
Here $\mathbf{y}$ is the observed outcome, in many cases called the ground truth.
$\mathbf{\hat{y}}$ is the prediction made by (\ref{OLS}). The goal is to find the
$\mathbf{\beta}$ that minimizes (\ref{MSE}). For this there is an alalytical solution 
as long as the matrix in (\ref{OLS}) is reversible. In other words: This can be solved
analytically as long as there are more datapoints than there are variables (features, inputs).
The solution to the equation can be written as 
\begin{equation}
\mathbf{\beta} = (\mathbf{X}^T\mathbf{X})^{-1}\mathbf{X}^{T}\mathbf{y}
\label{OLS solution}
\end{equation}
\subsection{Bias-Variance tradeoff}
\subsection{Neural Networks}
\subsection{Recurrent Neural Networks}
\subsection{Long- Short-Term Memory}
\subsection{Strategies for including static features}
\subsection{Drawbacks}
