\section{Rainfall-Runoff modelling}
\subsection{Physical modelling}
\subsection{Drawbacks}

\section{Machine Learning}
We give a brief explanation of the basics in Machine Learning here for better context 
before elaborating on the LSTM model central to this thesis.
The term Machine Learning was coined in \citationneeded by .... 
Machine Learning is a type of frequentist approach to statistical analysis where 
one creates a statistical model, often with several million parameters and finds 
the value of each parameter that makes the model approximate the data in the most 
accurate manner. How to find these parameters and how they are used differ for each 
model type. 
\subsection{Linear regression}
In the simple case of the Ordinary Least Squares (OLS) model we have a model on the form 
\begin{equation}
\mathbf{\hat{y}} = \mathbf{\beta} \mathbf{X}
\label{OLS}
\end{equation}
This assumes that the outcome $\mathbf{\hat{y}}$ can we represented as a linear combination 
of some fitted parameters $\mathbf{\beta}$ and the input features $\mathbf{X}$.
The goal here is then to find the minimum of the mean squared error (MSE) of this.
The MSE is defined as 
\begin{equation}
MSE = |\mathbf{y} - \mathbf{\hat{y}}|^2
\label{MSE}
\end{equation}
Here $\mathbf{y}$ is the observed outcome, in many cases called the ground truth.
$\mathbf{\hat{y}}$ is the prediction made by (\ref{OLS}). The goal is to find the
$\mathbf{\beta}$ that minimizes (\ref{MSE}). For this there is an alalytical solution 
as long as the matrix in (\ref{OLS}) is reversible. In other words: This can be solved
analytically as long as there are more datapoints than there are variables (features, inputs).
The solution to the equation can be written as 
\begin{equation}
\mathbf{\beta} = (\mathbf{X}^T\mathbf{X})^{-1}\mathbf{X}^{T}\mathbf{y}
\label{OLS solution}
\end{equation}
\subsection{Bias-Variance tradeoff}
When training any kind of machine learning model, one usually divides the data 
into at least two parts: The training dataset and the testing dataset. The training 
dataset is for training a given model, while the testing dataset is to be kept separate 
from the training process so as to not make the performance metrics of the model too optimistic. 
By "optimistic" what is meant is that the error, for scalar values (\ref{MSE}) is 
lower on the data the model is trained on than on data the model has not seen under 
training. The function that is minimized under machine learning is called a 
cost function and while (\ref{MSE}) is very commonly used for scalar outcomes 
there exist many other cost functions all with different characteristics. \citationneeded

To explain 
why this is important we need to have a quick look at what is known as the 
bias variance tradeoff.
In the case of the OLS model the MSE can be rewritten into three parts:
\begin{equation}
    MSE = \text{Bias}^2 + \text{Variance} + \sigma^2
    \label{Bias Variance Decomp}
\end{equation}
For a full derivation of this, see \cite{vijayakumar2007bias}. When selecting and 
configuring machine learning models this tradeoff is essential. The following is 
a qualitative explanation of what each term in (\ref{Bias Variance Decomp})
represents:
\begin{itemize}
\item Bias: The bias is the part of the error that comes from a model's lack of complexity.  If one were to try and represent a non-linear system on the form of (\ref{OLS}) for instance one would struggle to model the more complex interactions between input and outcome.
\item Variance: In many ways this error is the opposite of bias. It comes from a given model having too much complexity. This could come from the model having too many parameters to train compared to how much data is available for training. 
\item $\sigma^2$: This is known as the irreducible error. It is the inherent error in the data that is used for training. The model cannot reduce the error below this value as it is indepentent from the model. To reduce this error one would have to gather more accurate data using better instruments for instance. 
\end{itemize}
The aim of training a machine learning model is to find the best tradeoff between model
complexity and stability in search of the minimum of (\ref{Bias Variance Decomp}).
\subsection{Neural Networks}
\subsubsection{Gradient Descent}
\subsection{Recurrent Neural Networks}
\subsection{Long Short-Term Memory}
\label{LSTM Theory}
\subsection{Feature selection}
\subsubsection{Permutation test}
One of the critisisms of machine learning methods
is that they are not easily interpretable. This 
is especially true if one wants to train a model 
on a dataset with an overwhelming amount of 
features. There are several strategies for 
selecting a smaller subset of features in a dataset.
The method we briefly describe here is called the 
permutation test:
Given a feature $j$, the permutation importance 
$i_j$ is equal to 
\begin{equation}
i_j = s - \frac{1}{K} \sum_{k=1}^K s_{k,j}\quad \cite{permutation}.
\label{Permutation equation}
\end{equation}
Here $K$ denotes how many permutations we average over for each feature, $s$ is
the model's score on the original data and $s_{k,j}$ is the score of permutation 
number $k$ of the feature $j$. In essence (\ref{Permutation equation}) describes 
how much the performance of a model varies when "destroying" the information 
contained in a feature, therefore explaining the importance of the feature 
according to the current model. It is then important to remember that this is not 
the true importance of the feature, only the importance the model thinks the feature 
has. The scoring method $s$ can be any model scoring statistic, often the $R^2$
score for regression.
A major problem for this method is that (\ref{Permutation equation}) could give 
unrealistically low significance to features that are highly correlated to other 
features. In this case the feature may very well be important, but the information 
contained in it is also contained in one or more other correlated features, meaning 
the model doesn't lose as much information as one may think.

A way around this is to first remove unneeded correlated features so that the model 
contains as little duplicate information as possible. Reducing the amount of 
features also has the added benefit of reducing the model variance because of 
decreased complexity and improving model interpretability as getting an intuitive 
understanding of what a model needs to properly approximate a system is easier when 
there are fewer features.

A method to remove static feature that is fairly simple to use is clustering on 
Spearman rank-oder correlations \cite{permutation_clustering}.
\subsection{Drawbacks}
