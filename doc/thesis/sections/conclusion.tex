\section{Summary}
We have shown that LSTM models perform well on CAMELS-GB as well as CAMELS, and 
also that a single model can perform well on both datasets when trained on them at the same time, even 
with less than optimal preprocessing of the mixed dataset.
In all cases models trained and validated on subsets of the same dataset perform 
approximately equally during cross validation and on the test sets. 
The models struggle to model the highest peaks and lowest valleys of streamflow, 
however. 

Training on CAMELS and validating on CAMELS-GB and vice-versa yielded disappointing, 
though not surprising results. We deemed this to be because the datasets cover 
domains that are too different from one another, as well as lacking equivalent 
attributes.

Our static attribute analysis yielded no obvious ways to improve the performance 
of process-driven models, as our models preferred static attributes containing 
information already used by process-driven models. Better performing 
machine learning models which are able to extract more information from the time 
series alone could perhaps alleviate this in the future, by emphasising on more 
meaningful attributes than those derived directly from the time series. 

From the fact (shown first by \citet{lstm_third_paper} and recreated in this thesis) 
that LSTM models outperform traditional models at generalization on 
CAMELS ($0.15$ median NSE improvement over the NWM), as well as the fact that we 
have shown that they perform well on CAMELS-GB, 
we believe it is possible to claim that machine learning models have the potential 
to vastly improve the understanding of the physics behind hydrological models. 
In the future we hope this leads to both improved physical understanding, 
as well as model quality, in several fields of the physical sciences. Hydrological 
systems are highly complex, and the need to use large scale data analysis to 
obtain a better understanding of the underlying physical processes has yet to be 
fulfilled. 

\section{Future work}
In the previous chapters we mentioned several ways to improve the results of this 
thesis. Here we give a brief summary.

One of the simplest ways to improve the performance of a machine learning model is to give 
it more and higher quality training data. To our knowledge there now exists three 
additional CAMELS-like datasets we have not used in this dataset \citep{CAMELS_CL, CAMELS_AU, CAMELS_BR}.
Training LSTM models on these datasets, as well as attempting to combine them with 
CAMELS and CAMELS-GB is something we believe could improve the generalizability of 
machine learning models trained for hydrological modelling.

There are likely several ways to improve the performance of our machine learning 
model through modifying the model architecture. A simple modification to start with 
could be to implement an LSTM based model with a one-dimensional CNN input layer. 
This is intended to reduce the number of time steps needed, thus reducing the 
computational cost of training the model.

Another way to improve the performance of a machine learning model is to improve 
the hyperparameter tuning. This could be especially important 
for transfer learning, as the reduction of variance is often important for 
generalization \citep{elemstatlearn}. This could be a more manageable task if 
the aforementioned CNN layer is implemented to decrease training time.

A more experimental approach would be to implement a physics based penalty term 
in the cost function. A simple example of 
this is described in the previous chapter. This could improve the interpretability 
of an LSTM and also increase the consistency of the output, as put by \citet{hybrid_paper}.

A similar, but more advanced approach than stated above, is to employ an LSTM as 
part of a traditional model in a hybrid approach. This could lead to better generalizability and 
interpretability. There are several ways to do this, one approach could be to implement 
an LSTM instead of a given process in a process-driven model. This could 
then be used to compare to the output of said process in the ordinary model versus 
the hybrid model. This is likely very computationally expensive, as the process-driven 
models are more expensive than pure LSTM models. A simpler hybrid model discussed 
in this thesis is a setup where the LSTM is trained to correct the error of 
the output of a traditional model. This way one could leverage the already existing 
model benchmarks on CAMELS \citep{CAMELS_hydroshare}.

\begin{comment}
One should also look into the possibility of implementing LSTM models in Statkraft's 
Shyft\citep{Shyft}, making it much easier to employ these models on many datasets, 
as well as greatly simplifying the process of comparing with other models.
\end{comment}
