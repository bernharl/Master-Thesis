Today's process-driven hydrological models struggle to accurately model the complex physical 
systems the field of catchment hydrology consists of. 
As others before us, we exploit newly released large-sample datasets as well as the increasing utility 
of Machine Learning models. 
We employ LSTM models on time series from two datasets in four configurations. 
We show that LSTM models are able to generalize and make satisfactory predictions on ungauged basins, also when 
trained on two different datasets at the same time. 
In addition to this, we rank the basin attributes using permutation feature importance, 
based on the four data configurations in this thesis. 
The attributes deemed most important by the LSTM models are mostly attributes derived 
directly from the time series the model already has access to. This indicates that 
there is potential to improve the long term memory of our machine learning models.
Our hope is that improving this in the future could lead to attribute rankings 
that indicate which attributes could be important for use in existing process-driven 
models. This either through adding new processes to existing models, or by guiding the 
tuning of model parameters through observable data instead of optimization schemes.
