Today's process-driven hydrological models struggle to accurately model the complex 
large scale physical systems the field of catchment hydrology consists of. 
As others before us, we exploit newly released large-sample datasets that combine 
hydrological time series and static basin attributes.
We show that LSTM models are able to generalize and make satisfactory predictions 
on ungauged basins, also when trained on two different datasets at the same time. 
An LSTM trained on CAMELS and CAMELS-GB datasets achieves a median validation NSE of 0.66 
and 0.78 respectively, compared to the NWM, which scores a median of ~0.55 on CAMELS alone.
We currently know no publicly available NWM benchmarks performed on CAMELS-GB.
In addition to this, we rank the basin attributes using permutation feature importance.
The attributes deemed most important by the LSTM models are mostly attributes derived 
directly from the time series the model already has access to. This indicates that 
there is potential to improve the long term memory of our machine learning models.
Our hope is that improving this in the future could lead to attribute rankings 
that indicate which attributes could be important for use in existing process-driven 
models. This either through adding new processes to existing models, or by guiding the 
tuning of model parameters through observable data instead of optimization schemes.
