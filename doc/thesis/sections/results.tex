In this chapter we present the results relevant to our discussed topics in the 
next chapter. To keep this chapter structured in a way that makes it easy to 
look up we divide our results into five main sections presenting model importance 
and feature importance for four cases:
\begin{enumerate}
    \item Models trained on CAMELS-GB \citep{CAMELS_GB}.
    \item Models trained on CAMELS \citep{CAMELS_US} in additiion to traditional 
        models provided by \citationneeded.
    \item Models trained on a dataset comprised of both CAMELS and CAMELS-GB.
    \item Models trained on CAMELS and validated on CAMELS-GB and vice-versa.
    \item Models refit on the full train set and valiated on the test set not 
        touched in sections 1 - 4.
\end{enumerate}

\section{Models trained on CAMELS-GB}
In this section we present the results related to model selection and feature 
importance when training and predicting on CAMELS-GB \citep{CAMELS_GB}. To act 
as a proof of concept of the feature importance method described in Chapter 
\ref{Feature selection} we also include the performance and feature importance of 
a model trained using dataset a (see Table \ref{attribute table}) in addition to
static attributes directly derived from the observed outcome.

\subsection{Performance}
\begin{figure}
    \centering
    \includegraphics{{results_section/camels_gb/cdf_val}.pdf}
    \caption{Cumulative distribution function of the NSE score of LSTM models trained 
    on CAMELS-GB \citep{CAMELS_GB}. "Overfit model" is a model deliberately trained 
    using static basin attributes derived from the runoff time series of the basins. 
    The other models are described in Table \ref{all models}.}
    \label{CAMELS-GB CDF validation}
\end{figure}
Figure \ref{CAMELS-GB CDF validation} shows the performance of several LSTM models 
trained on CAMELS-GB \citep{CAMELS_GB}. The overfit model is trained on dataset a 
in Table \ref{attribute table} in addition to attributes directly derived from the 
observed outcome.
The overfit model significantly outperforms all other models.
 Of the other models there seems to be generally two levels of 
performance. The models using attribute subsets a and b perform similarly while 
the models trained with no attributes in general perform worse, with or 
without dropout.
There seems to be little overall difference in performance between EA-LSTM and 
ordinary LSTM models.
 All ungauged  models perform with a negative NSE value for 
approximately 10\% of the basins.

\subsection{Importance}
\section{Models trained on CAMELS}
In this section we present the results related to model selection and feature 
importance when training and predicting on CAMELS \citep{CAMELS_US}. This section 
is meant to show whether we are able to produce results similar to the those of 
\citet{lstm_third_paper} in a way that is comparable to the rest of our experiments.
In addition we present apparent feature importance of these trained models.
\subsection{Performance}
\begin{figure}
    \centering
    \includegraphics{{results_section/camels_us/cdf_val}.pdf}
    \caption{Cumulative distribution function of the NSE score of models trained 
    on CAMELS \citep{CAMELS_US}. "VIC" is the Variable Infiltration Capacity model 
    calibrated on CAMELS. "SAC-SMA" is the SACramento Soil Moisture Accounting 
    model calibrated on CAMELS. 
    These benchmarks are provided by \citet{CAMELS_hydroshare} and are originally 
    created by \citet{VICbench} and are trained per-basin as opposed to the other 
    models.
    "NWM" is a benchmark of the National Water Model run on CAMELS. (CHECK IF PER-BASIN, 
    I THINK ITS UNGAUGED!!!). This benchmark is available without any licensing at 
    \citet{NWMbench} and we use an already preprocessed version provided by 
    \citet{lstm_third_paper}.
    The other models are described in Table \ref{all models} and are based on the 
    models originally trained by \citet{lstm_third_paper} but retrained for better 
    comparison with the other LSTM models in this thesis.}
    \label{CAMELS-US CDF validation}
\end{figure}
\subsection{Importance}
\section{Models trained on CAMELS and CAMELS-GB}
\subsection{Performance}
\begin{figure}
    \centering
    \includegraphics{{results_section/mixed/cdf_val}.pdf}
    \caption{Cumulative distribution function of the NSE score of LSTM models trained 
    on a dataset consisting of both CAMELS \citet{CAMELS_US} and CAMELS-GB \citep{CAMELS_GB}. 
    The models are described in Table \ref{all models}. The top figure shows the 
    performance of the models on CAMELS-GB and the bottom figure shows the performance 
    on CAMELS.}
    \label{mixed CDF validation}
\end{figure}
\subsection{Importance}
\section{Models trained for transfer learning between CAMELS and CAMELS-GB}
\subsection{Performance}
\begin{figure}
    \centering
    \includegraphics{{results_section/transfer/cdf_val}.pdf}
    \caption{Cumulative distribution function of the NSE score of LSTM models trained 
    on CAMELS \citep{CAMELS_US} and validated on a validation part of CAMELS as well as 
    the entirety of CAMELS-GB \citep{CAMELS_GB}. The top figure shows the 
    performance of the models on CAMELS-GB and the bottom figure shows the performance 
    on CAMELS.}
    \label{transfer CDF validation}
\end{figure}
\subsection{Importance}
\section{Test performance of best models}
Note that when doing model selection we do not look at test set performance, only 
performance on the cross validated train set.
%\begin{table}
%    \begin{adjustbox}{width=\textwidth}
%    \begin{tabular}{lr}
\toprule
{} &     Coeff \\
\midrule
Q95                   &  1.211018 \\
elev\_50               &  0.464968 \\
conductivity\_cosby\_50 &  0.443736 \\
intercept             &  0.443277 \\
porosity\_cosby\_50     &  0.409635 \\
conductivity\_hypres\_5 &  0.257509 \\
high\_prec\_dur         &  0.211253 \\
gauge\_easting         & -0.211231 \\
tawc                  & -0.326509 \\
elev\_max              & -0.349975 \\
conductivity\_cosby\_5  & -0.351292 \\
baseflow\_index\_ceh    & -0.417237 \\
bares\_perc            & -0.463398 \\
elev\_10               & -0.496951 \\
conductivity\_cosby\_95 & -0.839606 \\
porosity\_hypres\_5     & -1.030274 \\
low\_prec\_freq         & -1.055273 \\
dwood\_perc            & -1.288324 \\
p\_mean                & -2.015090 \\
ewood\_perc            & -2.146395 \\
urban\_perc            & -2.332153 \\
shrub\_perc            & -3.735298 \\
grass\_perc            & -4.010000 \\
crop\_perc             & -4.868455 \\
\bottomrule
\end{tabular}


%    \end{adjustbox}
%    \caption{Table for attempt at linear regression. This model is fit on the static
%    features as input values, and the NSE of an LSTM trained without static features
%    for each basin in the validation set. The $R^2$ score of this model is $\approx 0.5$}
%    \label{linreg_no_static_table}
%\end{table}
%In table \ref{linreg_no_static_table} we can see that our linear regression used to model 
%the relationship between NSE value of models and static features has decided that 
%several features are important. All these coefficients have a p-value $<0.5$. The 
%problem with this model was that it achieved an $R^2$ score of $\approx 0.5$, meaning 
%it does not actually explain the relationship well. We therefore choose to change 
%strategy for feature selection. 
\begin{comment}
\begin{figure}
    \includegraphics{{correlation_reduction/all_features/corr_and_dendrogram}.pdf}
    \caption{The Pearson correlation matrix and the corresponding hierarchial
    cluster of the training set before removing correlated features. The process
    of creating the dendrogram is described in chapter \ref{Feature selection}. 
    For readability's sake we cannot include every label in the correlation matrix.}
    \label{corr_matrix_full}
\end{figure}

\begin{figure}
    \includegraphics{reduced_matrix.pdf}
    \caption{Correlation matrix after backward selection}
    \label{corr_matrix_reduced}
\end{figure}

%\subsection{Performance of full model}
\begin{figure}
%\includegraphics[scale=1]{{permutation/all_features_cv/histogram_all}.pdf}
    \includegraphics{{figures/permutation/all_features_cv/histogram_all}.pdf}
\caption{The two most important (above) and least important (below) features according 
to the permutation test}
\label{Hist all}
\end{figure}



\begin{figure}
    \centering
    \includegraphics{{CDFs/mixed_model_comparisons}.pdf}
    \caption{top kjeks}
    \label{main cdf}
\end{figure}
\end{comment}
