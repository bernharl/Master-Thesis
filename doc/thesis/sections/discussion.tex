In this chapter we discuss the performance of our models and compare to similar 
works. The chapter is split into two main sections: In the first section we 
discuss the performance of our machine learning models, compare to other similar 
experiments discuss ways to improve the machine learning aspects of this thesis.
In the second section we discuss the physical implications we believe the success 
of these machine learning models could have. We also discuss possible implications 
of our attribute importance rankings. 
\section{Machine Learning}
In this section we discuss the machine learning aspects of the results in this thesis.
We analyze the pure performance of our models and discuss whether we believe it 
is possible to improve said performance.
\subsection{Model Selection}
\label{discuss model selection}
Section \ref{CAMELS-GB results}-\ref{Mixed results} 
all agree that performance is increased when introducing static attributes. In 
addition to this, we see that the ordinary LSTM models that treat the static 
attributes as time series are the highest performing models. We find it likely that 
this is due  to the decrease in model complexity in an EA-LSTM compared to an LSTM. 
For example, Mixed$_\text{ea-lstm}$ has 205057 trainable parameters (weights and 
biases), Mixed$_\text{lstm}$ has 285953 and Mixed$_\text{none}$ has 266497. The 
EA-LSTM is actually a less complex model than an ordinary LSTM trained without 
static attributes. The EA-LSTM's ability to come close to the performance of an 
LSTM using $~72\%$ of the trainable parameters is notable, however. 
\subsection{Overfitting}
There are no clear signs of overfitting in our selected models. Table 
\ref{results summary table} indicate 
similar performance between the cross validated models and the refit models 
validated on the test set. There are some discrepancies in the CDF plot of 
the performance of the models trained on CAMELS 
shown in Figure \ref{test cdf} (second from top), leading to a median NSE of 
0.65 on the test set as opposed to 0.68 in the cross validated case. This small 
deviation could perhaps be explained by the findings of \citet{lstm_second_paper},
 which found that an ensemble of LSTM models performs better than a singular LSTM 
 because of nontrivial differences in performance caused by randomness in the 
 training process. For computational reasons we have chosen not to use model 
 ensembles anywhere in this thesis.

 What we can view as overfitting is the case of transfer learning. Figure 
 \ref{training progress transfer} indicates that Transfer$_\text{GB, lstm, none}$ 
 and Transfer$_\text{US, lstm, none}$ start overfitting on their respective training 
 datasets. This overfitting isn't apparent when validating on basins from the 
 same dataset as they are trained on, but very quickly manifests when validating 
 on the other dataset. We view it as likely that if CAMELS-GB and CAMELS were to 
 have a perfectly overlapping supply of input data, it would show that this is 
 still the case for this case as well. This could indicate that training LSTM models 
 for transfer learning between different datasets either needs improved pre-processing, 
 or the introduction of some kind of regularization. 
\subsection{Hyperparameter Tuning}
We do not spend significant time in this thesis tuning hyperparameters. Instead, 
we chose to use the hyperparameters derived by \citet{lstm_second_paper}. The one 
exception to this is in the case shown in Section \ref{CAMELS-GB results}. We found 
that using dropout in this case at best makes no difference and at worst slightly 
decreases performance. In some cases we also tested whether increasing the 
sequence length from 270 days to 365 days increases performance, but our results 
indicate that this leads to no performance difference. The lower sequence length, 
the lower memory requirement per batch size, so we decided to keep using 270 days. 

We acknowledge that it may be possible to further increase the performance on 
CAMELS-GB by doing a thorough hyperparameter tuning experiment, this is 
something that definitely should be done in the future. 
\section{Physical Aspects}
In this section we discuss the physical implications of our results and what could 
possibly be used in future process-driven models and potential hybrid process-driven 
and machine learning models.
\subsection{Static Attribute Analysis}
\label{discuss static attributes}
As the results presented in Section \ref{CAMELS-GB results} and \ref{CAMELS results} 
indicate that the most important attributes are merely attributes derived from the 
input time series, it is likely that we cannot extract much new information from 
the attribute rankings. Other ways to rank attributes should also be attempted 
in the future, however. \citet{lstm_second_paper} used a robustness test which 
involves gradually adding Gaussian noise to a feature and checking how much it 
affects model performance (DOUBLE CHECK THIS!!!). \citet{OrigCAMELSRanking} 
uses a different method involving blah blah. These two methods should also be 
attempted on CAMELS-GB in addition to the permutation test employed in this 
thesis.

\subsection{On the feasibility of transfer learning}
The results in Section \ref{Transfer learning section} show that the models trained 
for transfer learning do not benefit from adding static attributes. We also see 
that the case of US$\rightarrow$GB significantly improves the opposite. This makes 
sense as Great Britain has much smaller climatic variance than the United States 
\citationneeded. We believe these two phenomena could be related. It is likely 
that the relatively lower performance from using static attributes comes from the 
fact that the datasets have different climatic conditions. There could also be 
room for improvement in how we preprocess both the attributes and the time series. 
As seen in Table \ref{transfer table} we assume that the temperature time series 
of CAMELS behave like symmetrical periodic functions, therefore leading to 
\begin{equation}
T_\text{average} \approx  \frac{T_\text{min}+T_\text{max}}{2}.
\end{equation}
Ideally, we need to have temperature as a time series for a given model to be able 
to properly model snow and frozen soil accumulation, which is very important for 
hydrologic models \citationneeded. 
\subsection{Comparison to traditional models}
Confirming the results of \citet{lstm_second_paper,lstm_third_paper}, our models 
trained on CAMELS performs significantly better than traditional models. In this 
thesis we compare to two process-driven models: VIC and NWM, and one conceptual 
model: SAC-SMA. 

What is new in this thesis is that models trained on both CAMELS and CAMELS-GB 
also perform better on CAMELS than the addressed traditional models. Our hypothesis 
(CHECK IF OTHERS ALSO CLAIM THIS) \citationneeded is that this is due to the LSTM 
models being better at modelling basins with special conditions leading to 
a rainfall-runoff ratio different from 1. When the rainfall-runoff ratio is lower 
than one it usually implies that the rainfall is either stored as snow or frozen 
soil, or that it is vaporized through photosynthesis in local vegetation \citationneeded ??. 

