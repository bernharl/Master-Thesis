In this chapter we briefly describe the two datasets used to get all results in 
this thesis. We introduce the CAMELS dataset \citep{CAMELS_US} first as it is the 
older dataset and is used as the basis for the analysis of 
\citet{lstm_first_paper, lstm_second_paper, lstm_third_paper}, and then CAMELS-GB 
\cite{CAMELS_GB}.
\section{The CAMELS dataset}
The Catchment Attributes and Meteorology for Large-sample Studies (CAMELS) dataset \citep{CAMELS_US}
is a dataset compiled from several previous hydrological datasets in an attempt to 
improve the availability of data for large-scale hydrological modelling. It includes 
data for 671 basins across the United States, with time series spanning from 
approximately the year of 1980 to 2008.

We choose to focus on the time series data 
used by \citet{lstm_second_paper}. This selection of data consists of an extension 
of the original Maurer dataset contained in CAMELS provided by \citet{maurer_hydroshare}.
The time series provided by the extender Maurer dataset that we use as inputs are 
the following:
\begin{itemize}
    \item Precipitation [mm/day]
    \item Shortwave Radiation [W/m$^2$]
    \item Maximum air temperature [C\degree]
    \item Minimum air temperature [C\degree]
    \item Water vapor pressure [Pa]
\end{itemize}
The streamflow data corresponding to the meteorological forcing data mentioned above 
stems from the United States Geological Survey (USGS). It contains streamflow for 
all the basins in the dataset.

In addition to time series, CAMELS contains several static basin attributes describing 
location and topography, climatic indices, hydrological signatures, soil characteristics, geological 
characteristics and land cover characteristics for each basin.
Attributes describing location and topography are derived from the USGS part of 
the N15 dataset and are basin-wise averaged. The attributes of interest here are the topographic ones, as 
location is not beneficial to physical modelling \footnote{Though it could in theory 
improve the performance of a statistical model}.
The climatic indices are directly derived from the time series provided by Daymet, 
another forcing time series dataset contained in CAMELS. The climatic index
\textbf{\texttt{frac\_snow}} is derived from the fraction of rainfall during 
subzero temperature, for instance. 
The hydrological signatures are derived from the streamflow data provided by 
USGS. These attributes contain info such as the average daily discharge 
\textbf{\texttt{q\_mean}}.
The land cover statistics describe land cover such as the forest fraction and 
the dominant land cover class. There is only a land cover fraction for the dominant 
land cover class for each basin. The land cover data is derived from the 
Moderate Resolution Imaging Spectroradiometer (MODIS) data, 
except for \textbf{\texttt{forest\_frac}}, which is derived from USGS data. MODIS 
data comes from satellites, making it possible to gather equivalent data from 
other regions of the world without having to perform on-site measurements.
The soil characteristics describe the soil properties of each basin in the 
dataset. These attributes are derived from \citet{SoilData}. There are two 
attributes for soil depth. One is from the paper just mentioned, and one is defined 
differently and stems from \citet{pelletier}. In this thesis we use the attribute 
\textbf{\texttt{soil\_depth\_pelletier}}, as it stems from the same data source as 
in CAMELS-GB \citep{CAMELS_GB}.

Complementing the CAMELS dataset are several hydrological model benchmarks provided 
by \citet{CAMELS_hydroshare}. 
These are calibrated and run on the time series data contained in CAMELS or the data 
it is derived from. In this thesis we include the benchmark of the VIC model we 
briefly describe in Section \cite{VIC}. Additionally \citet{lstm_third_paper} used 
and provided a preprocessed  benchmark of the NWM \citep{NWMbench, NWMbench-paper}. 
This benchmark is known as the 
NOAA (National Oceanic and Atmospheric Administration) NWM reanalysis. The benchmark 
is released with no license and can as of April 2021  be found at 
\url{https://registry.opendata.aws/nwm-archive/}.
The preprocessed version is available on \citet{lstm_third_paper}'s 
Github page: \url{https://github.com/kratzert/lstm_for_pub/blob/master/data/nwm/nwm_daily.pkl}.


\section{The CAMELS-GB dataset}
As per the Open Government License v3 the dataset is distributed under, we are required 
to include the statement "Contains data supplied by Natural Environment Research Council."
The CAMELS-GB dataset \citep{CAMELS_GB} is inspired by the CAMELS dataset and is 
aimed to be the equivalent to the CAMELS dataset for basins in Great Britain, 
as opposed to American basins in the CAMELS dataset. The dataset is the same in size,
also containing 671 basins. The time series span from 1970-2015, though, making 
it span roughly 15 more years than the CAMELS dataset.

The forcing time series we use in this thesis are
\begin{itemize}
    \item Precipitation [mm/day]
    \item Average temperature [C\degree]
    \item Wind speed [m/s]
    \item Humidity [g/kg]
    \item Shortwave radiation [W/m$^2$]
    \item Longwave radiation [W/m$^2$]
\end{itemize}
In addition there are two more time series which we exclude that describe potential 
evapotranspiration for well-watered grass. We exclude these as they are derived 
features and not measured, and they would make the comparison to CAMELS more difficult.

The static attributes in CAMELS-GB are structured similarly to those in CAMELS. 
There are location and topographic attributes, which describe the geographical 
location of each basin, as well as the giving a description of the topography. 
As stated above, the geographical attributes are not of interest to a physical model, 
and are therefore not used here. The topography attributes are provided by 
\citet{morris1990digital}.
The climatic indices are derived from the forcing time series, similarly to CAMELS. 
They contain average precipitation, aridity, snow fraction and other attributes.
Hydrologic signatures are derived in the same manner, being processed information 
derived from the streamflow time series for each basin. 
The land cover attributes are provided by \citet{rowland2017land}. The classes 
have been derived using a random forest classifier on satellite images, meaning 
they suffer from uncertainty from the classification and the original data source. 
As opposed to CAMELS, there is land cover percentage for each class, not just the 
dominant class in each basin.
\textbf{\texttt{root\_depth}} is, like in CAMELS, taken from \citet{pelletier}. 
All other soil attributes are provided by \citet{hiederer2013mappinga, hiederer2013mappingb}.
The soil attributes are limited to the top 1.3 meters of soil. CAMELS-GB 
contains three attribute categories not present in CAMELS: hydrogeological attributes, 
hydrometry and discharge uncertainty, and human influence attributes. These three 
categories are not used in this thesis.

The meteorological time series and static attributes in CAMELS-GB are 
not all equivalent with CAMELS. This makes it difficult to combine the datasets
for a universal model, as one has to remove and/or modify a significant amount of 
static attributes to get overlapping datasets.
 
As far as we know there is no currently available traditional model benchmark dataset 
akin to \cite{CAMELS_hydroshare} for CAMELS-GB. The creation of such a dataset would 
likely be a challenge because of the time series contained in CAMELS-GB. There is for 
instance not enough information available to run VIC (see section \ref{VIC}) on 
the dataset.
