\section{The CAMELS dataset}
The Catchment Attributes and MEteorology for Large-sample Studies (CAMELS) dataset \cite{CAMELS_US}
is a dataset providing data for 671 catchments spanning the United States, each
containing timeseries describing every day from 1980-2008. While 
it is a rather large dataset, 
in this section we will only describe the relevant parts of it for this study. 
As mentioned earlier what interests us is modelling streamflow with additional 
static information for the catchment we are modelling. The interesting parts of
the CAMELS dataset to us are then:
\begin{itemize}
    \item Meteorological forcing timeseries for each catchment.
    \item Streamflow timeseries for each catchment.
    \item Static attributes for each catchment.
\end{itemize}
There are three meteorological datasets in CAMELS, the one that is studied in \cite{CAMELS_US}
and \cite{lstm_second_paper} is the data sourced from Daymet \cite{thornton2012daymet}
Daymet contains several timeseries, the ones used in \cite{lstm_second_paper} are:
\begin{itemize}
    \item Precipitation
    \item Shortwave Radiation
    \item Maximum air temperature
    \item Minimum air temperature
    \item Water vapor pressure
\end{itemize}
We also stick to these for easy comparison's sake.
The dataset also contains results from popular physical models used in hydrological 
modelling. This makes it easy to use the dataset for benchmarking without having 
to run the physical models ourselves. The models we compare to are \begin{Large}
FIND OUT NAMES OF MODELS, DESCRIBE THEM IN DETAIL IN THEORY!!
\end{Large}
The interesting part of the CAMELS dataset to us is the addition of the static 
features. These contain information not used in traditional, physical models. 
Kratzert et. al. showed in \cite{lstm_second_paper} \cite{lstm_third_paper} that
with the addition of these static features a purely data driven LSTM model generalizes
better than traditional physical models. For more details on what we mean by 
"generalizes", see \ref{theory}. For a further describtion of the static features
we use in this thesis, see \ref{appendix: features camels us}.
For each day of forcing data, each catchment (with some erros) also contains 
streamflow data. This is what is used as the output for models.

\section{The CAMELS-GB dataset}
The CAMELS-GB dataset \cite{CAMELS_GB} is inspired by the CAMELS dataset and is 
essentially an attempt to replicating the CAMELS dataset on catchments in Great Britain, 
as opposed to American basins in the CAMELS dataset. The dataset is the same in size,
also containing 671 basins. The timeseries span from 1970-2015, though, making 
it span roughly 15 more years than the CAMELS dataset.
 The meteorological timeseries 
in this dataset however differ, so we cannot use the same models for both datasets
although that would be interesting to do in the search for better generalization.
The ones we use in this study are:
\begin{itemize}
    \item Precipitation
    \item Temperature
    \item Windspeed
    \item Humidity
    \item Shorwave radiation
    \item Longwave radiation
\end{itemize}
The dataset contains no physical model benchmarks, so we are a bit more limited 
with how we can measure the physical information lacking from traditional models.
It is still important for the assessment of LSTM-based models to show that it works
well on more than just one dataset, however. 
The CAMELS-GB dataset contains several static features, 76 of which are investigated
in this thesis.
