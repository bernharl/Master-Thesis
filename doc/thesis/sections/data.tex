In this chapter we briefly describe the two datasets used to get all results in 
this thesis. We introduce the CAMELS dataset \citep{CAMELS_US} first as it is the 
older dataset and is used as the basis for the analysis of \citet{lstm_first_paper, lstm_second_paper, lstm_third_paper}.
\section{The CAMELS dataset}
The Catchment Attributes and Meteorology for Large-sample Studies (CAMELS) dataset \citep{CAMELS_US}
is a dataset compiled from several previous hydrological datasets in an attempt to 
improve the availability of data for large-scale hydrological modelling. It includes 
data for 671 basins across the United States. 

As CAMELS consists of several datasets, we choose to focus on the time series data 
used by \citet{lstm_second_paper}. This selection of data consists of an extension 
of the original Maurer dataset contained in CAMELS provided by \citet{maurer_hydroshare}.
The time series provided by the extender Maurer dataset that we use as inputs are 
the following:
\begin{itemize}
    \item Precipitation [mm/day]
    \item Shortwave Radiation [W/m$^2$]
    \item Maximum air temperature [C\degree]
    \item Minimum air temperature [C\degree]
    \item Water vapor pressure [Pa]
\end{itemize}

The streamflow data corresponding to the meteorological forcing data mentioned above 
stems from \citet{usgs_streamflow}.

In addition to time series CAMELS contains several static basin attributes describing 
hydrological, geological and geographical properties of each basin. It has been 
shown that including these basins improves the performance and generalization of 
statistical models such as LSTMs \cite{lstm_second_paper}. 

Complementing the CAMELS dataset are several hydrological models provided by \citet{CAMELS_hydroshare}. 
These are calibrated and run on the time series data contained in CAMELS or the data 
it is derived from. In this thesis we include the benchmark of the VIC model we 
briefly describe in Section \cite{VIC}. Additionally \citet{lstm_third_paper} used 
and provided a preprocessed  benchmark of the NWM (see Section \ref{NWM}). 
This benchmark is known as the 
NOAA (National Oceanic and Atmospheric Administration) NWM reanalysis. The benchmark 
is released with no license and can as of April 2021  be found at 
\url{https://registry.opendata.aws/nwm-archive/}.
The preprocessed version is available on said paper's 
Github page: \url{https://github.com/kratzert/lstm_for_pub/blob/master/data/nwm/nwm_daily.pkl}.


\section{The CAMELS-GB dataset}
As per the Open Government License v3 the dataset is distributed under we are required 
to include the statement "Contains data supplied by Natural Environment Research Council."
The CAMELS-GB dataset \citep{CAMELS_GB} is inspired by the CAMELS dataset and is 
aimed to be the equivalent to the CAMELS dataset for basins in Great Britain, 
as opposed to American basins in the CAMELS dataset. The dataset is the same in size,
also containing 671 basins. The time series span from 1970-2015, though, making 
it span roughly 15 more years than the CAMELS dataset.
 The meteorological time series in addition to the catchment attributes are however 
 not all equivalent with CAMELS, so it is unlikely to get an optimal universal model 
 for both datasets as we need to exclude a significant amount of data from both 
 datasets to make them overlap.
The time series we use in this thesis are
\begin{itemize}
    \item Precipitation [mm/day]
    \item Average temperature [C\degree]
    \item Wind speed [m/s]
    \item Humidity [g/kg]
    \item Shortwave radiation [W/m$^2$]
    \item Longwave radiation [W/m$^2$]
\end{itemize}
In addition there two more time series which we exclude that describe potential 
evapotranspiration for well-watered grass. We exclude these as they are derived 
features and not measured.


As far as we know there is no currently available traditional model benchmark dataset 
akin to \cite{CAMELS_hydroshare} for CAMELS-GB. The creation of such a dataset would 
likely be a challenge because of the time series contained in CAMELS-GB. There is for 
instance not enough information available to run VIC (see section \ref{VIC}) on 
the dataset.
