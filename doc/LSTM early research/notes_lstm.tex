
% REMEMBER TO SET LANGUAGE!
\documentclass[a4paper,12pt,english]{article}
\usepackage{gensymb}
\usepackage[utf8]{inputenc}
\usepackage[a4paper, vmargin=0.75in, hmargin=0.7in]{geometry}

% Standard stuff
\usepackage{amsmath,graphicx,varioref,verbatim,amsfonts}
% colors in text
\usepackage[usenames,dvipsnames,svgnames,table]{xcolor}
% Hyper refs
\usepackage[colorlinks]{hyperref}
\usepackage{float}
\usepackage{cite}
% Document formatting
\setlength{\parindent}{0mm}
\setlength{\parskip}{1.5mm}
\usepackage{tikz}
%Color scheme for listings
\usepackage{textcomp}
\definecolor{listinggray}{gray}{0.9}
\definecolor{lbcolor}{rgb}{0.9,0.9,0.9}

%Listings configuration
\usepackage{listings}
\usepackage{subcaption}
\usepackage{physics}
%Hvis du bruker noe annet enn python, endre det her for å få riktig highlighting.
\lstset{
	backgroundcolor=\color{lbcolor},
	tabsize=4,
	rulecolor=,
	language=python,
        basicstyle=\scriptsize,
        upquote=true,
        aboveskip={1.5\baselineskip},
        columns=fixed,
	numbers=left,
        showstringspaces=false,
        extendedchars=true,
        breaklines=true,
        prebreak = \raisebox{0ex}[0ex][0ex]{\ensuremath{\hookleftarrow}},
        frame=single,
        showtabs=false,
        showspaces=false,
        showstringspaces=false,
        identifierstyle=\ttfamily,
        keywordstyle=\color[rgb]{0,0,1},
        commentstyle=\color[rgb]{0.133,0.545,0.133},
        stringstyle=\color[rgb]{0.627,0.126,0.941}
        }




%opening
\title{LSTM research}
\author{Bernhard Nornes Lotsberg}

\begin{document}
\maketitle
\section*{LSTM}
\begin{itemize}
\item RNN with long and short term memory.
\item Long term memory not possible in ordinary RNN's because of vanishing gradients.
\item Long term memory is very important for runoff modelling, as some features take a long time to impact. Snow for example.
\end{itemize}

\section*{EA-LSTM}
\begin{itemize}
\item Paper \cite{lstm_second_paper} introduces the EA-LSTM (Entity Aware), which is a modified version of the LSTM designed to better with this this problem.
\item Beneficial for hydrological modelling because it is able to process information about the current catchment that it is modelling (?). 
\end{itemize}

\section*{Usage of existing code}
\begin{itemize}
\item Notebook tutorial for using LSTM used in paper \cite{lstm_first_paper} can be found at \url{https://github.com/kratzert/pangeo_lstm_example}
\begin{itemize}
\item This repository includes a binder link for running the notebook in browser.
\item Notebook shows easy to understand examples of how to use Pytorch and how to load the CAMELS dataset.
\item Only the first experiment in the paper.
\end{itemize}
\item Code for paper \cite{lstm_second_paper} can be found at \url{https://github.com/kratzert/ealstm_regional_modeling}
\begin{itemize}
\item This github page contains links to datasets used as well as pre trained models. 
\item Need stronger computer to recreate the models, but should hopefully be possible to use the pre trained models from laptop.
\item Runs fine on my home desktop.
\item All results in the paper and all relevant code is here, including pre-trained models and code for creating and training models yourself.
\end{itemize}
\item Code for paper \cite{lstm_third_paper} can be found at \url{https://github.com/kratzert/lstm_for_pub}
\begin{itemize}
\item Includes step-by-step guide on how to recreate results from article. (Including bash scripts for automation)
\item Need Matlab for a few plots, but unsure if this is important
\end{itemize}
\item Spend most time on the first paper in the beginning, it is important to actually understand how an LSTM model works (not to mention RNNs in general!).
\end{itemize}

\section*{Purely data driven model}
\begin{itemize}
\item A lot of code already exists
\item First thing to try is to explicitly provide snow data (which is not done by papers cited in this document).
\item Possibly unreliable as we have less physical intuition, but unsure if a hybrid model performs better.
\item LSTM or EA-LSTM
\item There is room for improvement in a basic manner as well, as the papers state that a proper hyperparameter search hasn't been performed. This is probably time consuming and boring, though.
\end{itemize}

\section*{Hybrid model}
\begin{itemize}
\item Could be more intuitive, possibly also more reliable.
\item Trained using Shyft in the training process. Using the output from Shyft as an input for instance.
\item Probably more difficult to train.
\item LSTM or EA-LSTM
\end{itemize}

\section*{Choice of model}
\begin{itemize}
\item I think starting with a purely data driven model and then try introducing output from Shyft along with data not used by Shyft.
\item Several configurations need to be tested.
\item Before any of this I need to actually learn how to use Pytorch.
\item Paper \citep{lstm_third_paper} suggests that physical constraints applied to the LSTM models could improve results.
\item Hybrid model should at least be attempted!
\end{itemize}

\bibliographystyle{IEEEtran}
\bibliography{references_notes_lstm}
\end{document}

