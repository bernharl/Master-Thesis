
% REMEMBER TO SET LANGUAGE!
\documentclass[a4paper,12pt,english]{article}
\usepackage{gensymb}
\usepackage[utf8]{inputenc}
\usepackage[a4paper, vmargin=0.75in, hmargin=0.7in]{geometry}

% Standard stuff
\usepackage{amsmath,graphicx,varioref,verbatim,amsfonts}
% colors in text
\usepackage[usenames,dvipsnames,svgnames,table]{xcolor}
% Hyper refs
\usepackage[colorlinks]{hyperref}
\usepackage{float}
\usepackage{cite}
% Document formatting
\setlength{\parindent}{0mm}
\setlength{\parskip}{1.5mm}
\usepackage{tikz}
%Color scheme for listings
\usepackage{textcomp}
\definecolor{listinggray}{gray}{0.9}
\definecolor{lbcolor}{rgb}{0.9,0.9,0.9}

%Listings configuration
\usepackage{listings}
\usepackage{subcaption}
\usepackage{physics}
%Hvis du bruker noe annet enn python, endre det her for å få riktig highlighting.
\lstset{
	backgroundcolor=\color{lbcolor},
	tabsize=4,
	rulecolor=,
	language=python,
        basicstyle=\scriptsize,
        upquote=true,
        aboveskip={1.5\baselineskip},
        columns=fixed,
	numbers=left,
        showstringspaces=false,
        extendedchars=true,
        breaklines=true,
        prebreak = \raisebox{0ex}[0ex][0ex]{\ensuremath{\hookleftarrow}},
        frame=single,
        showtabs=false,
        showspaces=false,
        showstringspaces=false,
        identifierstyle=\ttfamily,
        keywordstyle=\color[rgb]{0,0,1},
        commentstyle=\color[rgb]{0.133,0.545,0.133},
        stringstyle=\color[rgb]{0.627,0.126,0.941}
        }




%opening
\title{LSTM research}
\author{Bernhard Nornes Lotsberg}

\begin{document}
\maketitle
\section*{LSTM}
\begin{itemize}
\item RNN with long and short term memory
\item wow
\end{itemize}

\section*{Usage of existing code}
\begin{itemize}
\item Notebook tutorial for using LSTM used in paper \cite{lstm_first_paper} can be found at \url{https://github.com/kratzert/pangeo_lstm_example}
\item Code for paper \cite{lstm_second_paper} can be found at \url{https://github.com/kratzert/ealstm_regional_modeling}
\item Code for paper \cite{lstm_third_paper} can be found at \url{https://github.com/kratzert/lstm_for_pub}
\item Spend most time on the first paper in the beginning, it is important to actually understand how an LSTM model works (not to mention RNNs in general!).
\end{itemize}

\bibliographystyle{IEEEtran}
\bibliography{references_notes_lstm}
\end{document}

